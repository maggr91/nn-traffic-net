\section{Actividades Involucradas}

\subsection{An\'{a}lisis de modelos actuales}


	Como parte inicial, es necesario analizar modelos que se han implementado
hasta la fecha sobre la realizaci\'{o}n de sistemas de control de tr\'{a}nsito
basados en redes neuronales. Cabe destacar que si bien existen diferentes
modelos,  se buscar\'{a} abarcar los que se
relacionen con el algoritmo de entrenamiento denominado \textit{Backpropagation} o \textit{propagaci\'{o}n
hacia atr\'{a}s} o el modelo Hopfield, empleando los otros como referencias
para tomar ideas de implementaci\'{o}n. Para realizar esto, se contemplan las
siguientes actividades:

\begin{enumerate}
  \item \textbf{Analizar las t\'{e}cnicas empleadas para implementar la red:}
  corresponde a entender la forma que otros investigadores han desarrollado
  la red, es decir, qu\'{e} acciones fueron consideradas para obtener un
  desempe\~{n}o {o}ptimo para la red, siendo importante la forma en
  que han logrado la toma de decisiones dentro de una red de
  sem\'{a}foros.
  
  \item \textbf{Buscar y listar factores considerados:} como se mencion\'{o}
  anteriormente, los factores y sus respectivos niveles afectan el desempe\~{n}o
  de la red, por lo cual es necesario obtener una lista adicional a la
  estipulada en los objetivos, las cual servir\'{a} de gu\'{i}a para descartar o
  si es posible agregar otros factores o las condiciones bajo las cuales se
  deber\'{a}n presentar los que se poseen actualmente.
  
  \item \textbf{Evaluar lista de factores:} una vez que se han obtenido los
  factores considerados en otras investigaciones, se proceder� a evaluar cuales
  pueden ser aplicables al ambiente vivido en Costa Rica.
  
  \item \textbf{Analizar el modelo de funcionamiento del sistema usado por el
  MOPT:} de forma que se pueda entender las pol\'{i}ticas que se emplean para
  administrar los tiempos de los sem\'{a}foros, as\'{i} como el an\'{a}lisis
  empleado en el centro de control para los ajustes o modificaciones del mismo.
  Como parte inicial se obtendr\'{a}n los registros de tiempos de duraci\'{o}n
  de las luces, con los ajustes que estos hayan realizado para poder visualizar
  si los resultados generados en un inicio son correctos y en caso de no serlo,
  determinar el grado de variaci\'{o}n seg\'{u}n los cambios realizados por los
  operadores del sistema.
  
  \item \textbf{Obtener un dise\~{n}o base de la red:} una vez analizados los
  factores y la forma en que fueron dise\~{n}adas las redes en otros casos, se
  proceder\'{a} a definir un modelo base para la red donde se pueda sintetizar
  aspectos abarcados en los puntos anteriores.
\end{enumerate}

\subsection{Dise\~{n}o la red}
\subsubsection{Definici\'{o}n de los elementos de la red}
	
		Parte fundamental de la realizaci\'{o}n de la red, es necesario definir las
	entradas, salidas y el criterio para medir qu\'{e} tan buenos son los
	resultados generados por la red.
	
	\begin{enumerate}
	  \item \textbf{Entradas de la red:} Como entrada de la red, se emplear\'{a} un
	  conjunto de datos provenientes de sensores simulados dentro del programa,
	  estos representar\'{a}n una forma en la cual se comunicar\'{i}a la red con el
	  exterior, dicha entrada llegar\'{a} a la capa incial de la red, es decir las
	  neuronas de entrada, en forma de vector de valores generados por el o los
	  sensores con los que disponga cada sem\'{a}foro agente dentro del sistema
	  distribuido. Sumado a lo anterior, la red neuronal de cada sem\'{a}foro,
	  recibir\'{a} un vector adicional de tama\~{n}o din\'{a}mico donde se obtenga el
	  estado actual o decisi\'{o}n tomada de los sem\'{a}foros adyacentes.
	  
	  \item \textbf{Salidas de la red:} una vez tomada la decisi\'{o}n, la red
	  neuronal de cada sem\'{a}foro, generar\'{a} un vector indicando el estado al
	  que pas\'{o} el sem\'{a}foro y el tiempo seleccionado para el cambio de luz.
	  
	  \item \textbf{Criterio de validez para los resultados:} para validar los
	  resultados obtenidos por la red, se tomar\'{a} como valor m\'{i}nimo esperado
	  los resultados generados por el sistema actual. El valor real esperado
	  ser\'{a} buscado por medio de tiempos de espera de los veh\'{i}culos
	  
	\end{enumerate}

\subsubsection{Creaci\'{o}n del modelo de red}

	Una vez determinados los puntos anteriores, se procede a la creaci\'{o}n de la
red y la implementaci\'{o}n del algoritmo de entrenamiento. Para este caso el
algoritmo a emplear corresponde a backpropagation de acuerdo a las revisiones de
referencias encontradas.

\subsection{Creaci\'{o}n programa de simulaci\'{o}n}
	
	Una vez establecido el modelo de la red, se proceder\'{a} a implementarlo
	mediante un lenguaje de programaci\'{o}n junto con este se incluir\'{a} un
	programa de simulaci\'{o}n el cual servir\'{a} para probar el algoritmo
	empleado en la red y determinar las decisiones necesarias durante cada prueba.

\subsection{Simulaci\'{o}n de los tiempos de espera}

	Por motivos del funcionamiento del sistema de control de tr\'{a}fico del MOPT,
este no genera o lleva registros sobre los tiempos de espera para los
veh\'{i}culos de acuerdo a los ajustes que se hayan realizado. Por lo anterior,
se requiere realizar simulaciones que permitan determinar estos tiempos haciendo
uso de los tiempos de cambios de luces que proporcione el Centro de Control de
Tr\'{a}fico del MOPT. Con esto, se simular\'{a} la llegada de los autos a los
sem\'{a}foros y el tiempo que empleen estos para poder pasar.

\subsection{Simulaci\'{o}n y entrenamiento la red}

	De la lista de factores seleccionada en puntos anteriores, se proceder\'{a} a
simular cada uno de estos, tomando en cuenta una serie de niveles en los que
estos se pueden presentar, cabe destacar que la zona a simular es limitada al
rango de avenida: 2, 4 y 8,  y las calles: 12, 10, 8 y 6 de San Jos\'{e}.

	Dicha simulaci\'{o}n se va a realizar\'{a} por medio del programa establecido
anteriormente, tomando como par\'{a}metros de medici\'{o}n los datos previamente
recolectados. Dentro de cada prueba generada, los resultados se comparar\'{a}n
con los registros del actual sistema de control de tr\'{a}nsito y de las
simulaciones realizadas a los tiempos de espera, de forma que se tenga un punto
de referencia para saber si la red est\'{a} funcionando de la forma esperada o si requiere ajustes logrando con esto entrenar a la red para
que corrija de la mejor forma posible, los errores que haya cometido.

\subsection{Iteraci\'{o}n de simulaciones}

	Consiste en la repetici\'{o}n del punto anterior, resaltando el hecho de hacer
	comparaciones para determinar el grado de error cometido por la red. Cabe
	destacar que entre cada iteraci\'{o}n se debe determinar el \textit{criterio
	de validez de los datos} con base en el cual se sabr\'{a} si las salidas son
	las mejores para cada caso.

\subsection{Evaluaci\'{o}n de ambientes vividos y factores de San Jos\'{e}}
	
	Por medio de an\'{a}lisis estad\'{i}stico se proceder\'{a} a contrastar los
resultados obtenidos por el uso de la red contra los datos reales. De forma que
se pueda obtener las diferencias a favor o en contra del uso de la red neuronal
para la administraci\'{o}n de los tiempos de espera en sem\'{a}foros y de los
ajustes de tiempos para cambios de luces.




