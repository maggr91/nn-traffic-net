\section{Alcance y Limitaciones}
		
		
		Esta tesis se enfocar\'{a} en las situaciones ocurridas \'{u}nicamente en San
		Jos\'{e} dentro de las zonas en las que opera actualmente el sistema de control de
	tr\'{a}fico, primero en la ciudad de San Jos\'{e} en las \textbf{avenida:} 2, 4
	y 8,  y las \textbf{calles:} 12, 10, 8 y 6; que ser\'{a} tomada como base inicial para el
	desarrollo y entrenamiento de la red neuronal.
	
		Adicionalmente, se elaborar\'{a} una red neuronal que funcione en un m\'{i}nimo de
	seis sem\'{a}foros ubicados en diferentes intersecciones de la zona delimitada
	anteriormente. Estos tomar\'{a}n las decisiones en conjunto para administrar y
	coordinar los tiempos que deben permanecer en luz verde o cuando es necesario
	que uno o varios de estos cambien a luz roja. Se realizar\'{a} varias
	simulaciones ya que se busca determinar bajo qu\'{e} factores funciona mejor el
	sistema mencionado, para esto, los datos que se obtengan como resultado del
	funcionamiento de la red ser\'{a}n valorados con respecto a las bit\'{a}coras
	que sean proporcionadas por el \textit{Centro de Control de Tr\'{a}fico del
	MOPT}.
	Cabe destacar que s\'{o}lo se proceder\'{a} a hacer pruebas en otras regiones de San Jos\'{e} hasta obtener resultados confiables de las pruebas realizadas en las zonas descritas.
	
		Como parte de las labores requeridas, se crear\'{a} un programa para simular los
	escenarios planteados para entrenamiento y prueba de la red neuronal. No se
	tomar\'{a}n como par\'{a}metro de entrada el consumo de combustible de los autom\'{o}viles,
	ni los gastos relacionados con estos. De igual forma, no se validar\'{a}n las
	entradas que reciba la red, es decir, todos los datos presentados a la red
	ser\'{a}n correctos sin consideraciones de alg\'{u}n tipo que afecte a las
	entradas de la red. Tampoco se contempla la implementaci\'{o}n y pruebas
	correspondientes en un ambiente real, esto debido a que el centro de control de
	tr\'{a}fico no dispone de recursos para invertir en la realizaci\'{o}n de esto.
	No obstante, la funcionalidad y resultados podr\'{a}n ser apreciados por medio del programa de simulaci\'{o}n mencionado, por lo que el esfuerzo se centra m\'{a}s en definir, desarrollar y probar el modelo de red neuronal que logre hacer su cometido de la mejor forma posible.
	
		El no contar con los sensores necesarios para la detecci\'{o}n de veh\'{i}culos
	automotores  as\'{i} como el presupuesto para adquirirlo, se convierte en una de
	las principales limitantes por las cuales el modelo no ser\'{a} implementado en un
	ambiente real. De igual forma, no se cuenta con el personal y tiempo necesarios para llevar a cabo esta labor, ya que el tiempo establecido para la realizaci\'{o}n de este trabajo podr\'{i}a no ser suficiente para llevar a cabo ambas labores, es decir, tanto el desarrollo del modelo como la implementaci\'{o}n en un ambiente real del mismo.
		
		La implementaci\'{o}n del modelo resultado de esta tesis se deja como tema para su
	posterior realizaci\'{o}n en otro trabajo, motivando a otras personas a retomar la
	soluci\'{o}n y brindar sus propios aportes y mejoras al mismo.\\

\section{Lista de Entregables}

Como parte de realizaci\'{o}n de la tesis se realizar\'{a} un modelo alternativo
que utilice el algoritmo de propagaci\'{o}n hacia atr\'{a}s para su
entrenamiento y la documentaci\'{o}n correspondiente. Lo cual se divide en los
siguientes entregables:

\begin{itemize}
  \item Implementaci\'{o}n del algoritmo para la programaci\'{o}n de la red: En
  el lenguaje y plataformas especificados previamente. \'{E}ste consistir\'{a} de la
  l\'{o}gica de su implementaci\'{o}n, y los archivos en el lenguaje de
  programaci\'{o}n seleccionados.
  \item Programa de simulaci\'{o}n de escenarios para la red  neuronal: este
  programa ser\'{a} empleado para poder realizar pruebas a la red, de forma que
  se puedan realizar los ajustes necesarios para que esta cometa la menor
  cantidad de errores posibles. Para este se tomar\'{a}n en cuenta los factores
  que afectan el desempe\~{n}o de la red.
  \item Documentaci\'{o}n del an\'{a}lisis estad\'{i}stico de los resultados obtenidos al
  entrenar la red neuronal, de forma que se pueda determinar bajo cu\'{a}les
  condiciones esta funciona mejor. De igual forma se contempla una
  comparaci\'{o}n con respecto a los tiempos de cambios de luces obtenidos del
  sistema actualmente utilizado por el Centro de Control de Tr\'{a}fico del
  MOPT.
  \item Datos obtenidos de las pruebas realizadas a la red neuronal: estos datos
  corresponden a los resultados finales sobre los ajustes que fueron hechos
  sobre la red, los errores que se corrigieron y su evaluaci\'{o}n con respecto
  al criterio de validez para determinar los valores reales esperados.
\end{itemize}

\section{Beneficios Esperados}
	
	\subsection{La organizaci\'{o}n}
			Esta tesis ayudar\'{a} a poner en claro las mejoras que se dan en
	caso de lograr la implementaci\'{o}n de este modelo en las calles de San Jos\'{e}. Lo
	anterior se debe a la necesidad de realizar una serie de an\'{a}lisis estad\'{i}sticos
	para evaluar diferentes factores que inciden en el control del tr\'{a}fico, con
	la finalidad de poder saber ante qu\'{e} factores funciona bien, o no, el
	modelo propuesto para el sistema, pero que a la vez permitir\'{a} contrastar la
	informaci\'{o}n estad\'{i}stica obtenida del sistema actual permitiendo
	saber las deficiencias espec\'{i}ficas que existan del mismo.
	
		Por otro lado, al contar con las comparaciones mencionadas anteriormente, el
	Centro de control de tr\'{a}nsito del MOPT podr\'{a} hacer uso de esta
	retroalimentaci\'{o}n para llevar a cabo cualquier �ajuste� que sea necesario de
	forma que se busque mejorar las disminuciones de tiempo y poder lograr m\'{a}s del
	30\% que se ha logrado actualmente, as\'{i} como mejorar el \'{i}ndice de ahorros en
	combustible anuales superando los \$4 millones, es decir m\'{a}s de lo indicado
	en \cite{Mata2009} y trayendo consigo una mejor respuesta y una mayor
	satisfacci\'{o}n de los conductores.
	
	\subsection{El estudiante}
		Para el caso del investigador, se espera obtener conocimientos m\'{a}s amplios
	sobre las redes neuronales, buenas pr\'{a}cticas involucradas dentro de la
	realizaci\'{o}n de sistemas de control de tr\'{a}nsito en general y
	especialmente los que se basen o hagan uso de redes neuronales. Con esto se
	pretende llegar a obtener conocimientos mayores a los que haya dejado la
	carrera de Ingenier\'{i}a en Sistemas de Informaci\'{o}n y poder abarcar temas
	que exijan un aprendizaje, y no s\'{o}lo la puesta en pr\'{a}ctica de
	temas vistos durante la carrera. 

	\subsection{La Universidad}
		Por \'{u}ltimo, para la Universidad Nacional se busca incentivar la
	realizaci\'{o}n de tesis o proyectos que se inclinen m\'{a}s por la
	investigaci\'{o}n donde se involucren temas que no se basen en asuntos,
	soluciones comunes o se limiten a la realizaci\'{o}n de otro sistema. Se
	espera lograr que la misma universidad sea la encargada de motivar a los
	aspirantes a indagar sobre temas nuevos que resulten retadores por el grado de
	complejidad que implican, pero que al final dejen un gran aporte en diferentes
	\'{a}reas y que ayuden a obtener investigaciones de gran calidad.
	
