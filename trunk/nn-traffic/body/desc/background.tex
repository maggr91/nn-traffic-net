\section{Antecedentes}
	
	
		En Costa Rica a lo largo de los a\~{n}os se ha podido observar como la gran
	cantidad de veh\'{i}culos automotores saturan sus principales ciudades, en especial
	el caso de su capital. Para el 2012 y seg\'{u}n el peri\'{o}dico La Naci\'{o}n [13]
	alrededor de 19.000 buses y 260.000 autos luchan por un espacio para poder
	avanzar por las calles de San Jos\'{e}.
	
		Como parte de las medidas para mitigar problemas de embotellamientos, el
	Ministerio de Obras P\'{u}blicas y Transportes (MOPT) puso en funcionamiento el
	2007 [14] un sistema de sem\'{a}foros inteligentes el cual funcionaba para ese
	entonces en 180 intersecciones de la capital haciendo uso de unas 145
	c\'{a}maras aproximadamente. Sin embargo, en muchas ocasiones en las que se circul\'{o} por la capital, se pudo notar un problema en particular: en muchas situaciones un veh\'{i}culo no pod\'{i}a continuar avanzando, esto por el hecho de que el sem\'{a}foro de la siguiente intersecci\'{o}n estaba en rojo. A pesar de contar con dicho sistema inteligente se segu\'{i}an dando este tipo de problemas y para agravar la situaci\'{o}n, al pasar por estas calles no es posible encontrar alg\'{u}n tipo de sensor el cual le conceda la propiedad de inteligente a este sistema.
	
		A la luz de las noticias anteriores y siendo afectado por uno de los
	embotellamientos, se presta m\'{a}s atenci\'{o}n a una serie situaciones presentadas.
	Por ejemplo, puede darse el escenario en el cual hay autom\'{o}viles esperando a que el sem\'{a}foro pase a verde, al momento en que estos llegan a la siguiente intersecci\'{o}n el sem\'{a}foro de esta sigue en rojo, esto podr\'{i}a hacer pensar que el sem\'{a}foro deber\'{i}a estar este en verde justo antes de que lleguen, no obstante esto no ocurre siempre. Por otro lado en caso de estar esperando en una intersecci\'{o}n a que el sem\'{a}foro cambie, si no vienen autom\'{o}viles en el otro sentido, no deber\'{i}a mantenerse en luz roja y lo ideal ser\'{i}a que cambie a luz verde.
		
		Por \'{u}ltimo, cuando las luces de los sem\'{a}foros cambian en secuencia, aparentan
	saber que se aproximan carros hacia ellos y que por lo tanto deben realizar el
	cambio de luz, pero esto ocurre por coincidencias de los ciclos de luces
	asignados a cada sem\'{a}foro o son causa de una sincronizan llevada a cabo para realizar el cambio de luces de esta forma.

		Muchas de las interrogantes anteriores fueron contestadas luego de poder
	asistir a una exhibici\'{o}n del sistema de control de tr\'{a}nsito, localizado en las
	oficinas del Centro de Control de Tr\'{a}nsito del MOPT en San Jos\'{e}. Como resultado
	de esto qued\'{o} claro que no se trata de un sistema inteligente automatizado,
	sino de un sistema de control centralizado desde el cual los operadores del
	mismo pueden configurar los tiempos para los cambios de luces de los sem\'{a}foros dentro de la red.  A ra\'{i}z de la situaci\'{o}n encontrada, se formula la idea de encontrar una forma de mejorar esta situaci\'{o}n, de aqu\'{i} que se tome como base el manejo de sistemas de control de tr\'{a}fico por medio de redes neuronales dado que, tal como se mencion\'{o} anteriormente, las ventajas que estas brindan sobre otro tipos de sistemas como los sistemas expertos, son de gran beneficio sobretodo el tema en cuesti\'{o}n.
	
		Desde hace d\'{e}cadas a nivel mundial se han implementado diferentes sistemas
	para el manejo avanzado del tr\'{a}fico, desde se\~{n}ales para regular los l\'{i}mites de
	velocidad, hasta sem\'{a}foros automatizados para regular el flujo de veh\'{i}culos que
	pasan por las intersecciones. Conforme han cambiado las necesidades, se han
	requerido sistemas de sem\'{a}foros inteligentes los cuales no s\'{o}lo puedan tomar decisiones con respecto a los veh\'{i}culos automotores que se encuentran frente al sensor correspondiente, sino que tambi\'{e}n tengan una noci\'{o}n del entorno que los rodea.

			En el Georgia Tech Research Institute durante el 1993 [3], se realiz\'{o} una
	investigaci\'{o}n sobre aplicaciones para control de tr\'{a}fico con redes neuronales,
	en el cual se muestran diferentes escenarios, la aplicaci\'{o}n del modelo
	Hopfield para el control de los sem\'{a}foros y el uso de las redes neuronales para la previsi\'{o}n de las congestiones por medio del algoritmo de propagaci\'{o}n hacia atr\'{a}s. No obstante para este se tomaron como criterios las capacidades de cada segmento de las calles, los rangos de flujo y su potencial, as\'{i} como los efectos que tendr\'{i}a el cambio de una luz sobre \'{a}reas lejanas. Cabe destacar que en este trabajo se propone la realizaci\'{o}n de una funci\'{o}n \'{u}nicamente para sincronizar los sem\'{a}foros adyacentes, de forma que se beneficie el cambio realizado en uno de ellos con la finalidad de mejorar el flujo de tr\'{a}fico que se esta generando.

		En 2006, miembros de la IEEE, desarrollan un modelo basado en un sistema
	multiagente h\'{i}brido sin supervisi\'{o}n, usando como escenario secciones del
	distrito central de negocios de Singapur, demuestran resultados con mejoras de
	hasta el 78\% en reducci\'{o}n de atrasos. Este se basa en usar cada sem\'{a}foro como
	un agente empleando un h\'{i}brido entre red neuronal difusa, junto con algoritmos
	evolutivos [23].
	
		En el 2008 en una publicaci\'{o}n realizada por estudiantes de la Universidad de
	Sacramento [4], se plantean aspectos sobre el uso de redes neuronales
	artificiales para formar parte de un gran sistema denominado IDUCT o
	Intelligence Decision-making system for Urban Traffic-Control de su nombre en
	ingl\'{e}s. Dicho modelo consiste de siete elementos y uno de los cuales corresponden a redes neuronales como parte del sistema de decisi\'{o}n inteligente, espec\'{i}ficamente en la parte de aprendizaje.

		Finalmente, en Alemania durante la segunda mitad del 2010 [2], Stefan L�mmer
	de la Universidad de Tecnolog\'{i}a de Dresden y Dirk Helbing de ETH Zurich crearon un
	modelo computacional basado en las calles de Dresden para probar un sistema en
	el cual los sem\'{a}foros se comunicaban uno con otro para ajustar los tiempos
	en los que la luz verde deber\'{i}a permanecer encendida.
	
		Esta tesis busca definir el modelo para un sistema de sem\'{a}foros distribuidos
	el cual emplee redes neuronales para el aprendizaje, \'{u}nicamente mediante
	propagaci\'{o}n hacia atr\'{a}s de forma supervisada y se administren los tiempos de
	cambios en las luces de los sem\'{a}foros. Con este, a diferencia de los
	anteriores, se busca lograr una toma de decisiones donde se tomen en cuenta la informaci\'{o}n distribuida en la red de sem\'{a}foros, en este caso cada sem\'{a}foro realizar\'{a} su procesamiento individual con la diferencia de tomar como insumo par\'{a}metros provenientes de otros sem\'{a}foros que se vean afectados por este o que lo afecten, de tal forma que se pueda llegar a una respuesta que permita el flujo adecuado del tr\'{a}fico. Como parte de los par\'{a}metros de entrada para la red se tomar\'{a}n diferentes factores que incidan en su funcionamiento adecuado, no limit\'{a}ndose a las capacidades de las calles o de los tiempos que emplean los autom\'{o}viles, si no que agregando factores como obst\'{a}culos en la calle, o eventos que afecten el flujo normal del mismo.
	
		Como se ha podido notar, las soluciones por mejorar el control del tr\'{a}fico se
	han hecho notar en el campo de las redes neuronales, as\'{i} como en otro tipo de implementaciones como los algoritmos gen\'{e}ticos [8], ya sea como parte complementaria para el aprendizaje o como forma alternativa al uso de estas. Cabe destacar que la lista de ejemplos no acaba ah\'{i}, es posible encontrar otros ejemplos ya que existen diferentes formas de poder aplicar las redes neuronales, en especial en este tipo de soluciones.
