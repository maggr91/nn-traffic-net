%\chapter*{Introducci\'{o}n}
\chapter*{Introducci\'{o}n}\addcontentsline{toc}{chapter}{Introducci\'{o}n}
	\label{chap:introduction}
 
	%\section*{Introducci\'{o}n}

		 
		La incorporaci\'{o}n de nuevos veh\'{i}culos a la flota vehicular con la que cuenta
	actualmente el pa\'{i}s, ha causado un incremento en los tiempos requeridos para
	poder desplazarse en \'{a}reas altamente transitadas, como es el caso de la Gran
	\'{A}rea Metropolitana (GAM). Por otro lado, este problema trae consigo un aumento
	en la emisi\'{o}n de gases contaminantes. Seg\'{u}n estudios del Dr. Dobles
	\cite{Robles2011} , en Costa Rica, la principal fuente de emisi\'{o}n de gases que contribuyen con el efecto invernadero es el consumo de energ\'{i}a en el sector de transporte que consume combustibles importados derivados del petr\'{o}leo, que finalmente terminan afectando al medio ambiente y sobre todo a las personas que se encuentra dentro o en los alrededores de estas \'{a}reas v\'{i}ctimas de este congestionamiento.
	
		Frente a este tipo de eventos, en diferentes pa\'{i}ses se han implementado
	sistemas de control de tr\'{a}fico, los cuales incluyen una red de sem\'{a}foros que
	pueden ser controlados de forma remota desde un centro de control. La ciudad de
	New York fue una de las primeras en implementar un sistema de este tipo
	\cite{Greenman1998}, con un solo edificio localizado en Queens en las oficinas
	del Centro de Control de Tr\'{a}fico perteneciente al Departamento de tr\'{a}nsito.
	
		La idea de estos sistemas, es contar con c\'{a}maras o alg\'{u}n tipo de sensor que le
	permita a un operador, o al mismo sem\'{a}foro, obtener la informaci\'{o}n necesaria
	para poder tomar una decisi\'{o}n, as\'{i} como el registro de la misma para futuras
	labores que se vean afectadas por \'{e}stos.
	
		El lograr que los sem\'{a}foros tomen decisiones \textit{inteligentes}, se ha
	intentado lograr de diferentes formas con soluciones por medio de  algoritmos basados en Inteligencia
	artificial. Rajendra en su libro [9], describe como el objetivo de la
	Inteligencia Artificial (AI de sus siglas en ingl\'{e}s), es tratar de lograr que
	las computadoras de alguna forma realicen las labores en la que los humanos son
	buenos. La definici\'{o}n de inteligencia artificial tiende a variar entre los
	autores, no obstante Rajendra hace menci\'{o}n de una que deja bastante claro su significado:

		\begin{quote}
			\textit{``Inteligencia artificial es la parte de las ciencias de
			la computaci\'{o}n interesada en dise\~{n}ar sistemas de computaci\'{o}n inteligentes que exhiban las
			caracter\'{i}sticas que se asocian con la inteligencia en el comportamiento
			humano''} \cite{Rajendra2005} %[9] (Rajendra, 2005)
		\end{quote}

		Como parte adicional a del avance, se han incorporado mejoras a este tipo de
	sistemas para intentar optimizar el desempe\~{n}o aut\'{o}nomo de los mismos. De esta
	forma, las redes neuronales (ANN de sus siglas en ingl\'{e}s) han representado una
	alternativa para poder lograrlo. En pa\'{i}ses como Alemania se han implementado
	modelos en los cuales los sem\'{a}foros toman decisiones basados en su entorno
	y no \'{u}nicamente por los autos en una determinada carretera \cite{Ben2010}
	%[2].
