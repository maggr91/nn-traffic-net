\chapter{Plan de Trabajo}
	\label{chap:work}
	
	\section{Resultados y Productos Esperados}
	
	Esta secci\'{o}n abarca los resultados del entrenamiento de la red y el
an\'{a}lisis estad\'{i}stico posterior a esto. Junto a lo anterior, se listan
los productos que se esperan producir como parte del desarrollo de la
investigaci\'{o}n.
	
Resultados esperados comprenden:
	\begin{itemize}
	  \item Encontrar un modelo de red neuronal que permita mejorar los tiempos
	  empleados por los veh\'{i}culos y que se reduzcan los congestionamientos por
	  causa de su mala administraci\'{o}n.
	  \item Que la toma de decisiones de forma distribuida por parte de los
	  sem\'{a}foros logre evitar tiempos de espera innecesarios. 
	  \item Reducci\'{o}n de los errores cometidos por la red de forma que para
	  otro trabajo de investigaci\'{o}n pueda ser puesto a prueba en lugares como
	  Heredia, Alajuela y Cartago.
	  \item Lograr un margen de error aceptable el cual permita en un futuro pasar
	  a la implementaci\'{o}n en ambiente real de la red neuronal.
	\end{itemize}

Se espera producir:
	\begin{itemize}
	  \item Programa de simulaci\'{o}n para entrenamiento de la red.
	  causa de su mala administraci\'{o}n.
	  \item Algoritmo de programaci\'{o}n de la red.
	  \item An\'{a}lis estad\'{i}stico donde se demuestre el desempe\~{n}o logrado
	  por la red, y se contraste con los valores que se presentan actualmente por
	  el sistema de control de tr\'{a}fico del MOPT.
	\end{itemize}
	
	\section{Riesgos del proyecto}
	
	\begin{table}[!h]
			\centering
			\begin{tabular}{|p{1cm}|p{3cm}|p{2cm}|p{2cm}|p{2cm}|p{3cm}|}
				\hline
				\textbf{ID} & \textbf{Riesgo} & \textbf{Prob.} & \textbf{Impacto} & \textbf{Magnitud} & \textbf{Acciones de
				Administraci\'{o}n}\\ \hline 
				RNN-1 & No contar con los factores necesarios para probar el correcto
				funcionamiento de la red neuronal causando una falta de pruebas del
				desempe\~{n}o de la misma & 2 & 4 & 8 & Revisar de forma exhaustiva la lista
				de factores obtenida de los trabajos anteriores, y validarla de forma que se
				garantice una correcta selecci\'{o}n de los escenarios para pruebas.
				\\
				\hline
				RNN-2 & No contar con los datos estad\'{i}scos generados por el sistema de
				control de tr\'{a}fico del MOPT necesarios para contrastar los resultados
				generados por la red neuronal & 3 & 4 & 12 & Mantener contacto activo con
				el miembro del centro de control de tr\'{a}fico de forma que no se pierda
				la colaboraci\'{o}n mediante llamadas, correos o visitas al centro. En caso
				de materializarse se proceder\'{a} a realizar los datos estad\'{i}stico de
				forma manual\\ 
				\hline 				
				 
			\end{tabular}
			\caption{Riesgos}
			\label{tab:risk}
		\end{table}
		
	\section{Proyecci\'{o}n o cronograma}	