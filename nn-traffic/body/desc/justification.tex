\section{Justificaci\'{o}n}

		
		Siempre ha existido el inter\'{e}s por lograr una mejora en el tr\'{a}nsito vehicular
	y se puede notar como la incorporaci\'{o}n de inteligencia artificial dentro de esta
	\'{a}rea  se ha ido incrementando con el paso de los a\~{n}os. Costa Rica, al igual que muchos pa\'{i}ses a nivel mundial, ha dado los primeros pasos para poder lograr la transici\'{o}n a estas mejoras, si bien en algunos se han logrado implementar mejores opciones, esto no siempre ha conseguido satisfacer en forma constante las necesidades de los usuarios. Muchas quejas han sido presentadas por los conductores [17] y diferentes factores inciden dentro de este tipo de problemas, desde una simple lluvia hasta desviaciones por construcciones o arreglos que se realicen aumentando las concentraciones de veh\'{i}culos en ciertos puntos.
	
		El incorporar las redes neuronales a una red de sem\'{a}foros resulta un reto,
	pero a la vez con su implementaci\'{o}n exitosa se lograr\'{a}n grandes mejoras al sistema
	de tr\'{a}nsito vehicular de los pa\'{i}ses que opten por realizar dicho proyecto.
	
		En el modelo que se posee actualmente en Costa Rica, no se dispone de alguna
	forma en que estos sem\'{a}foros funcionen de forma aut\'{o}noma mientras que al hacer
	uso de redes neuronales se dar\'{a} un gran avance en la administraci\'{o}n de estos ya
	que, como se dijo anteriormente, la red neuronal le permitir\'{a} a los sem\'{a}foros aprender y con el tiempo poder hacer predicciones del flujo de autom\'{o}viles tal y como se plantea en [3].
	
		Actualmente un sem\'{a}foro perteneciente a un sistema inteligente b\'{a}sico puede
	decidir qu\'{e} hacer, pero esta decisi\'{o}n corresponde \'{u}nicamente a lo que este
	puede observar por medio de los sensores con los que cuente, por otro lado al
	encontrarse varios sem\'{a}foros interconectados y tomando decisiones seg\'{u}n los datos que uno le indique al otro, todos podr\'{a}n contar con un panorama m\'{a}s claro y amplio de lo que realmente est\'{a} sucediendo en un nivel global.
	
		En el \'{a}rea de las redes neuronales existen una gran cantidad de modelos
	utilizados para poder implementarlas, algunos de los cuales han sido objeto de
	estudio para formar parte de alguna propuesta de sistema para el control de
	tr\'{a}fico. La tesis busca definir y  proponer una alternativa a dichos modelos, el cual permita establecer un sistema de simulaci\'{o}n para la predicci\'{o}n que administre de forma id\'{o}nea el tiempo de esperas de los autom\'{o}viles en los sem\'{a}foros mientras circulan por las calles, tomando como forma de aprendizaje la propagaci\'{o}n hacia atr\'{a}s con entrenamiento de tipo supervisado mediante la cual se obtendr\'{a} una modificaci\'{o}n que se adapte lo mejor posible al problema en cuesti\'{o}n.
	
		Cabe destacar que al realizarse la simulaci\'{o}n, se estar\'{a}n probando todos los
	aspectos sobre el modelo de red neuronal a desarrollar por lo que los
	resultados, y efectos (positivos o negativos) generados se lograr\'{a}n visualizar
	de esta forma. Por esta raz\'{o}n el esfuerzo se ver\'{a} enfocado en llegar a obtener un modelo que logre hacer bien su cometido, pero su implementaci\'{o}n en un ambiente real no forma parte de esta tesis, si no que se deja abierto para una futura realizaci\'{o}n, la cual se ver\'{a} favorecida gracias al an\'{a}lisis estad\'{i}stico de los factores que realizar\'{a} como una forma de confirmar el funcionamiento adecuado de la red con respecto al sistema actual.
	
	\subsection{Beneficios}
	
	
		El principal aporte de este trabajo consiste en brindar una alternativa a los
	modelos actuales de sistemas de sem\'{a}foros inteligentes basados en redes
	neuronales con el que se pueda contrastar las ventajas y desventajas que
	proporcionan los actuales con respecto al modelo que se proponga en la tesis.
	
		Dicho modelo al ser una alternativa, ayudar\'{a} con ideas de modificaciones que
	son posibles realizar a los modelos actuales, en los que pueden no haberse
	considerado ciertos tipos de variables debido al hecho de no estar pensados espec\'{i}ficamente para este tipo de problemas. Con esto se podr\'{a}n abrir puertas a futuras mejoras del mismo o a posibles agregados que  no fueron considerados, por alg\'{u}n motivo, como parte de esta investigaci\'{o}n.
	
		De igual forma, esta tesis ayudar\'{a} a poner en claro las mejoras que se dan en
	caso de lograr la implementaci\'{o}n de este modelo en las calles de San Jos\'{e}. Lo
	anterior se debe a la necesidad de realizar una serie de an\'{a}lisis estad\'{i}sticos
	para evaluar diferentes factores que inciden en el control del tr\'{a}fico, con
	la finalidad de poder saber ante qu\'{e} factores funciona bien, o no el modelo
	propuesto para el sistema, pero que a la vez permitir\'{a} contrastar la
	informaci\'{o}n estad\'{i}stica obtenida del sistema actual permitiendo
	saber las deficiencias espec\'{i}ficas que existan del mismos.
	
		Finalmente, al contar con las comparaciones mencionadas anteriormente, el
	Centro de control de tr\'{a}nsito del MOPT podr\'{a} hacer uso de esta
	retroalimentaci\'{o}n para llevar a cabo cualquier �ajuste� que sea necesario de
	forma que se busquen mejorar las disminuciones de tiempo y poder lograr m\'{a}s del
	30\% que se ha logrado actualmente, as\'{i} como mejorar el \'{i}ndice de ahorros en
	combustible anuales superando los \$4 millones, es decir m\'{a}s de lo indicado
	en [15] y trayendo consigo una mejor respuesta y una mayor satisfacci\'{o}n de
	los conductores.
	
	
	
	
