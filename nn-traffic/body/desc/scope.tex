\section{Alcance y Limitaciones}
		
		
		Esta tesis se enfocar\'{a} a las situaciones ocurridas \'{u}nicamente en San Jos\'{e}
	dentro de las zonas en las que opera actualmente el sistema de control de
	tr\'{a}fico, primero en la ciudad de San Jos\'{e} en las \textbf{avenida:} 2, 4
	y 8,  y las \textbf{calles:} 12, 10, 8 y 6; que ser\'{a} tomada como base inicial para el
	desarrollo y entrenamiento de la red neuronal.
	
		Adicionalmente, se elaborar\'{a} una red neuronal que funcione en un m\'{i}nimo de
	seis sem\'{a}foros ubicados en diferentes intersecciones de la zona delimitada
	anteriormente. Estos tomar\'{a}n las decisiones en conjunto para administrar y
	coordinar los tiempos que deben permanecer en luz verde o cuando es necesario
	que uno o varios de estos cambien a luz roja. Se realizar\'{a} varias
	simulaciones ya que se busca determinar bajo qu\'{e} factores funciona mejor el
	sistema mencionado, para esto, los datos que se obtengan como resultado del
	funcionamiento de la red ser\'{a}n valorados con respecto a las bit\'{a}coras
	que sean proporcionadas por el \textit{Centro de Control de Tr\'{a}fico del
	MOPT}.
	Cabe destacar que s\'{o}lo se proceder\'{a} a hacer pruebas en otras regiones de San Jos\'{e} hasta obtener resultados confiables de las pruebas realizadas en las zonas descritas.
	
		Como parte de las labores requeridas, se crear\'{a} un programa para simular los
	escenarios planteados para entrenamiento y prueba de la red neuronal. No se
	tomar\'{a}n como par\'{a}metro de entrada el consumo de combustible de los autom\'{o}viles,
	ni los gastos relacionados con estos. De igual forma, no se validar\'{a}n las
	entradas que reciba la red, es decir, todos los datos presentados a la red ser\'{a}n correctos sin consideraciones de alg\'{u}n tipo que afecte a las entradas de la red. Tampoco se contempla la implementaci\'{o}n y pruebas correspondientes en un ambiente real, esto debido a que el centro de control de tr\'{a}fico no dispone de recursos para intervenir en la realizaci\'{o}n de esto. No obstante, la funcionalidad y resultados podr\'{a}n ser apreciados por medio del programa de simulaci\'{o}n mencionado, por lo que el esfuerzo se centra m\'{a}s en definir, desarrollar y probar el modelo de red neuronal que logre hacer su cometido de la mejor forma posible.
	
		El no contar con los sensores necesarios para la detecci\'{o}n de veh\'{i}culos
	automotores  as\'{i} como el presupuesto para adquirirlo, se convierte en una de
	las principales limitantes por las cuales el modelo no ser\'{a} implementado en un
	ambiente real. De igual forma, no se cuenta con el personal y tiempo necesarios para llevar a cabo esta labor, ya que el tiempo establecido para la realizaci\'{o}n de este trabajo podr\'{i}a no ser suficiente para llevar a cabo ambas labores, es decir, tanto el desarrollo del modelo como la implementaci\'{o}n en un ambiente real del mismo.
		
		La implementaci\'{o}n del modelo resultado de esta tesis se deja como tema para su
	posterior realizaci\'{o}n en otro trabajo, motivando a otras personas a retomar la
	soluci\'{o}n y brindar sus propios aportes y mejoras al mismo.\\

\section{Lista de Entregables}

Como parte de realizaci\'{o}n de la tesis se realizar\'{a}n los siguientes entregables:

\begin{itemize}
  \item Modelo alternativo basado en el modelo de propagaci\'{o}n hacia atr\'{a}s y la
  documentaci\'{o}n correspondiente.
  \item Algoritmo para la programaci\'{o}n de la red.
  \item Programa de simulaci\'{o}n de escenarios para la red  neuronal.
  \item Documentaci\'{o}n del an\'{a}lisis estad\'{i}stico de los resultados obtenidos al
  entrenar la red neuronal.
  \item Datos obtenidos de las pruebas realizadas a la red neuronal.
\end{itemize}