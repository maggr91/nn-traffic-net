\section{Hip\'{o}tesis}

\textit{El uso de t\'{e}cnicas de IA posibilita la realizaci\'{o}n de un sistema
de control de tr\'{a}nsito distribuido, donde se utilicen como parte de los par\'{a}metros datos
proporcionados por los mismos sem\'{a}foros y que permita administrar los tiempos de
espera empleados por los conductores y aumentar la fluidez del tr\'{a}fico.}

\section{Objetivos}
	\subsection{Objetivo General}
	
		Desarrollar y simular un modelo que permita la implementaci\'{o}n y an\'{a}lisis de un
	sistema distribuido de sem\'{a}foros inteligentes basado en redes neuronales
	como mecanismo de IA, para lograr un mejor control del tr\'{a}fico que fluye
	por las zonas m\'{a}s congestionadas de San Jos\'{e}, en el que se consideren factores que obstruyen o alteran el flujo continuo del mismo.
	
	\subsection{Objetivos Espec\'{i}ficos}
	\begin{enumerate}
	  \item Analizar los modelos actuales para la implementaci\'{o}n de sistemas de
	  sem\'{a}foros basados en redes neuronales, as\'{i} como el sistema empleado en Costa Rica, para establecer puntos de referencia sobre el funcionamiento y rendimientos, que ayuden en la especificaci\'{o}n del modelo a desarrollar.
	  
	  \item Dise�ar e implementar un modelo de red neuronal empleando
	  propagaci\'{o}n hacia atr\'{a}s (backpropagation), para el entrenamiento de la red siendo uno de los modelos propuestos actualmente para la implementaci\'{o}n de un sistema de sem\'{a}foros inteligentes basado en redes neuronales.
	  \item Simular escenarios cambiantes en los que se incluyan factores en sus
	  diferentes niveles que afecten el tr\'{a}fico, donde se consideren eventos t\'{i}picos (alto n\'{u}mero de veh\'{i}culos), as\'{i} como los poco frecuentes (colisiones, carros varados, presencia de lluvia, reparaciones de v\'{i}as) dentro del rango avenida: 2, 4 y 8,  y las calles: 12, 10, 8 y 6 de San Jos\'{e}.
	  \item Evaluar los ambientes vividos y factores que se presentan en las
	  principales zonas de San Jos\'{e}, por medio de an\'{a}lisis estad\'{i}stico para poder contrastar los resultados obtenidos por el uso de la red neuronal contra los datos reales que ha generado actualmente el sistema de control de tr\'{a}fico.
	  \item Llevar a cabo un proceso iterativo de simulaciones el cual permita
	  mejorar los c\'{a}lculos realizados por redes neuronales aplicado a los diferentes factores del experimento, tomando como base de comparaci\'{o}n los datos actuales generados por el sistema de control de tr\'{a}fico para lograr diferencias notables en la administraci\'{o}n de tiempos en los sem\'{a}foros.
	  
	\end{enumerate}
