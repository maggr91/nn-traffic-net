\section{Asunci\'{o}n}

Tal y como se plante\'{o} anteriormente, con esta tesis se buscar obtener
respuestas a la hip\'{o}tesis:

\begin{quote}
	\textit{El uso de redes neuronales posibilita la realizaci\'{o}n de un sistema
de control de tr\'{a}nsito distribuido, donde se utilicen como parte de los par\'{a}metros datos
proporcionados por los mismos sem\'{a}foros y que permita administrar los tiempos de
espera empleados por los conductores y aumentar la fluidez del tr\'{a}fico.}
\end{quote}

	Apartir de esta, se asume el hecho de que las redes neuronales permiten
desarrollar este tipo de sistemas para el control de tr\'{a}fico que sometido a
diferentes factores lleve a cabo el reconocimiento de patrones y una
administraci\'{o}n adecuada de una red de sem\'{a}foros.

\section{Enfoque de Investigac\'{o}n}

	
	Debido a que la tesis busca definir qu\'{e} tan bien se adec\'{u}a la
	hip\'{o}tesis planteada a la realidad objetiva, se dejan por fuera creencias o juicios
subjetivos al respecto. De igual forma, el tema a investigar requiere de la
realizaci\'{o}n de an\'{a}lisis estad\'{i}sticos que permitan poder comparar
los resultados obtenidos durante el proceso de aprendizaje y la realizaci\'{o}n
respectivas de las pruebas a la red neuronal contra los datos que actualmente
se obtienen como resultado del funcionamiento del sistema de control de
tr\'{a}fico del MOPT.
	
	Por los motivos mencionados, no resulta factible utilizar un enfoque
cualitativo para esta investigaci\'{o}n. Por lo cual se define el uso del
enfoque cuantitativo para la realizaci\'{o}n de la misma.

\section{Tipo de Investigac\'{o}n}


	Al tenerse el prop\'{o}sito de relacionar la forma en que afecta el uso de
redes neuronales a la realizaci\'{o}n de sistemas de control de tr\'{a}nsito,
los cuales administren de forma adecuada los tiempos de espera en una red
de sem\'{a}foros, se escoge realizar un estudio correlacional entre el uso de
redes neuornales y la administraci\'{o}n de estos tiempos.

	Lo que se busca es entender y saber c\'{o}mo se comporta un sistema de este
tipo al lograr la incorporaci\'{o}n de una red neuronal. Cabe destacar que
dentro de este \'{a}mbito influyen una serie de factores:

\begin{itemize}
  \item Eventos t\'{i}picos como alto n\'{u}mero de veh\'{i}culos
  \item Eventos poco frecuentes como colisiones, carros varados, presencia de
  lluvia y reparaciones de v\'{i}as.
\end{itemize} 

	Sumado a lo anterior, se encuentra el hecho de que cada uno de estos se puede
	presentar en un nivel diferente implicando una variaci\'{o}n que favorezca o
	perjudique el desempe\~{n}o logrado por la red neuronal.
	
	 Al tener efectos diferentes de acuerdo al nivel en que se presente cada
factor, es necesario la realizaci\'{o}n de experimentos donde se cambien los
 factores para medir su efecto en la red,
 es decir, el dise\~{n}o de la investigaci\'{o}n cuantitava a emplear
 corresponde a un \textit{dise\~{n}o experimental}.
 
 	Para este caso se pretende manipular de forma intencional las entradas de la
 red de forma que se puedan abarcar los escenarios mencionados. Todo esto se
 lleva a cabo con la finalidad de medir, por medio de an\'{a}lisis
 estad\'{i}sticos, los efectos generados por cada nivel de los factores que
 incurren en un sistema de tr\'{a}nsito. Si bien las variables a afectar
 corresponden a los factores que incurren en el flujo del tr\'{a}nsito, su
 an\'{a}lisis en conjunto permitir\'{a}n determinar la forma en que afecta la
 incorporaci\'{o}n de redes neuronales a estos sistemas.
 
  	Como requisito para realizar este tipo de investigaci\'{o}n, es necesario
contar con una validez interna de la situaci\'{o}n que se experimenta. En
otras palabras, se requiere saber el grado de confianza de que los resultados
de los experimentos realizados son correctos o v\'{a}lidos. Es por esto que
requiere definir un criterio el cual permita medir que tan precisos son los
resultados. Esto forma parte de las acciones que se han de realizar junto
con la definici\'{o}n de las entradas y salidas de la red, aspectos detallados
en los siguientes apartados.


 
	
	
	
	
