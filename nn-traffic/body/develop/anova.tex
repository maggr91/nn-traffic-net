\section{ANOVA: Simulaciones del CCT}

Como parte de la realizaci\'{o}n de an\'{a}lisis estad\'{i}stico para los resultados de las simulaciones llevadas a cabo en el Centro de Control de Tr\'{a}nsito del MOPT, se consider\'{o} necesaria la realizaci\'{o}n de un ANOVA o An\'{a}lisis de Varianza.

El modelo del experimento cuenta con 12 casos y para cada uno de los cuales se obtuvieron los datos referentes a: tiempo de viaje (travel time), demora de viaje (travel delay), tiempo detenido durante el viaje (travel stop time), tiempo para atravesar intersecci\'{o}n (turning time) y demora para atravesar intersecci\'{o}n (turning delay)

Para cada una de \'{e}stas variables se procedi\'{o} a realizar el respectivo gr\'{a}fico haciendo uso de las cinco repeticiones de la simulaci\'{o}n con las que contaba cada caso. Como resultado del proceso, se pudo observar un comportamiento entre poisson y binomial al hacerse la comparaci\'{o}n de los diferentes casos, de aqu\'{i} que se considerara la realizaci\'{o}n de un ANOVA para determinar si alguno de los casos representaba una diferencia en comparaci\'{o}n a los dem\'{a}s.

En este experimento, la variable categ\'{o}rica corresponde a los casos, de los cuales se cuentan con doce niveles con sus respectivos datos. Cabe mencionar que cada caso corresponde a la combinaci\'{o}n de diferentes factores evaluados en varios niveles, aspecto que de igual forma se compara m\'{a}s adelante en el documento.

\subsection{Hip\'{o}tesis}

\begin{description}
	\item \textbf{H0:} El caso (escenario) evaluado en las simulaciones NO afecta los tiempos de los veh\'{i}culos para circular por las calles

	\item \textbf{HA:} El caso (escenario) evaluado en las simulaciones afecta los tiempos de los veh\'{i}culos para circular por las calles.
\end{description}