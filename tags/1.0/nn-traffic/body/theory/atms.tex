\section{Sistemas de Transporte Inteligente}

	Ezell \cite{Ezell2010}  pone en claro el hecho de que las Tecnolog\'{i}as de la
	Informaci\'{o}n han permitido obtener avances en el control de los sistemas de transportes al
permitir que los elementos de este -�veh\'{i}culos, caminos, luces de
tr\'{a}fico, se\~{n}ales de mensajes, entre otras- se vuelvan inteligentes al emplear chips
embebidos, as\'{i} como sensores que les permitan comunicarse unos con otros por
medio del uso de tecnolog\'{i}as cableadas o inal\'{a}mbricas. Los sistemas de
transportes inteligentes o \textit{Intelligent Transportation Systems} (ITS) que se han implementado a nivel mundial, han tra\'{i}do consigo mejoras significativas al desempe\~{n}o del sistema de transporte, el cual abarca no s\'{o}lo reducci\'{o}n de congestiones sino que tambi\'{e}n mejoras a la seguridad.
	
	De acuerdo con el autor anterior, los ITS consisten de una gran gama de
tecnolog\'{i}as y aplicaciones. Estas aplicaciones se pueden clasificar en cinco
categor\'{i}as mostradas en el cuadro \ref{tab:catITS} localizado en la
p\'{a}gina  \pageref{tab:catITS}.

\begin{table}[tb!hp]
			\centering
			\begin{tabular}{|p{4cm}|p{11.80cm}|}
				\hline
				\textbf{Sistema} & \textbf{Descripci\'{o}n}\\ \hline
				Sistemas avanzados de informaci\'{o}n para viajeros & Brindan
				informaci\'{o}n en tiempo real sobre rutas transitables y horarios o las direcciones de circulaci\'{o}n, as\'{i} como avisos sobre congestiones, accidentes o condiciones del tiempo. \\ \hline 
				Sistemas avanzados de gesti\'{o}n de transporte & Constituido por dispositivos de control de tr\'{a}fico como lo son las se\~{n}ales de tr\'{a}fico, se\~{n}ales de mensajes cambiantes o Variable Message Sings (VMS) y centros de operaci\'{o}n de tr\'{a}fico. \\ \hline
				Sistemas de cobro de transportes & Dentro de este se encuentran dispositivos para el cobro en carreteras como los peajes electr\'{o}nico de cobro, sistemas basados en cuotas o sistemas de cobros por distancia recorrida. \\  \hline
				Sistemas Avanzados de transporte p�blico & En los cuales se involucran trenes o buses de forma que estos informen sus ubicaciones reales para que los pasajeros conozcan esto. \\ \hline
				Sistemas inteligentes de transportes completamente integrados & Corresponde
				a sistemas m\'{a}s complejos en los que se involucran tanto a los veh\'{i}culos que circulan como a las v\'{i}as por las que estos pasan. Lo que se logra con este tipo de sistemas es establecer comunicaciones entre los activos mencionados y otros como sensores y se\~{n}ales de tr\'{a}fico. \\
				\hline
			\end{tabular}
			\caption{Categor\'{i}as de sistemas ITS}
			\label{tab:catITS}
\end{table}
	
	
% \begin{itemize}
%   \item \textbf{Sistemas avanzados de informaci\'{o}n para viajeros}, que
%   brindan informaci\'{o}n en tiempo real sobre rutas transitables y horarios o las direcciones de circulaci\'{o}n, as\'{i} como avisos sobre congestiones, accidentes o condiciones del tiempo.
%   \item \textbf{Sistemas avanzados de gesti\'{o}n de transporte:} constituido
%   por dispositivos de control de tr\'{a}fico como lo son las se\~{n}ales de tr\'{a}fico, se\~{n}ales de mensajes cambiantes o Variable Message Sings (VMS) y centros de operaci\'{o}n de tr\'{a}fico.
%   \item \textbf{Sistemas de cobro de transportes:} dentro de este se encuentran
%   dispositivos para el cobro en carreteras como los peajes electr\'{o}nico de cobro, sistemas basados en cuotas o sistemas de cobros por distancia recorrida.
%   \item \textbf{Sistemas Avanzados de transporte p�blico:} en los cuales se
%   involucran trenes o buses de forma que estos informen sus ubicaciones reales para que los pasajeros conozcan esto.
%   \item \textbf{Sistemas inteligentes de transportes completamente integrados:}
%   este corresponde a sistemas m\'{a}s complejos en los que se involucran tanto a los veh\'{i}culos que circulan como a las v\'{i}as por las que estos pasan. Lo que se logra con este tipo de sistemas es establecer comunicaciones entre los activos mencionados y otros como sensores y se\~{n}ales de tr\'{a}fico.
% \end{itemize}

	Dentro de estas categor\'{i}as, los Sistemas avanzados de
gesti\'{o}n de transporte o \textit{Active Transportation Management System}
abarcan parte de los sistemas relacionados con el control de se\~{n}ales de tr\'{a}fico. Su objetivo
principal es maximizar la productividad y eficiencia del sistema de transporte
mediante una administraci\'{o}n adecuada y pertinente de la infraestructura con
la que dispone la zona de cobertura logrando movilidad y accesibilidad.
\cite{Turnbull}

	Gracias a estos sistemas, se logran beneficios en diferentes formas debido a
que la ATM (Advanced Transportation Management)  ayuda a reducir la probabilidad de
accidentes, anunciar condiciones de las v\'{i}as,
direccionar el tr\'{a}fico y planeamiento del mismo \cite{Brinckerhoff2010}.
Algunas de las t\'{e}cnicas empleadas para esto se muestran en el cuadro
\ref{tab:technique} localizado en la p\'{a}gina  \pageref{tab:technique} 


\begin{table}[tb!h]
			\centering
			\begin{tabular}{|p{4cm}|p{11.80cm}|}
				\hline
				\textbf{T\'{e}cnica} & \textbf{Descripci\'{o}n}\\ \hline
				Armonizaci\'{o}n de velocidad & Consistiendo de ajustes
  din\'{a}micos a los l\'{i}mites de velocidad en las autopistas bas\'{a}ndose en los niveles de congesti\'{o}n. \\ \hline
				Advertencias de filas & Corresponde al uso de se\~{n}ales capaces de
  cambiar mensajes denominadas \textit{Variable Message Sings} (VMS).
  Permitiendo escribir notificaciones sobre posibles congestiones o cierres de carreteras de modo que los conductores puedan escoger rutas alternas. \\ \hline
				Direccionamiento din\'{a}mico & Similar al anterior, con la
  variante de presentar a los conductores las gu\'{i}as a rutas alternas. \\
  \hline
			Sistemas de control de luces de tr\'{a}fico	& Manejo de los
  cambios de luces de los sem\'{a}foros de acuerdo con las necesidades. \\
				\hline
			\end{tabular}
			\caption{T\'{e}cnicas empleadas por ATM}
			\label{tab:technique}
\end{table}


% \begin{itemize}
%   \item \textbf{Armonizaci\'{o}n de velocidad:} consistiendo de ajustes
%   din\'{a}micos a los l\'{i}mites de velocidad en las autopistas bas\'{a}ndose en los niveles de congesti\'{o}n.
%   \item \textbf{Advertencias de filas:} corresponde al uso de se\~{n}ales capaces de
%   cambiar mensajes denominadas \textit{Variable Message Sings} (VMS).
%   Permitiendo escribir notificaciones sobre posibles congestiones o cierres de carreteras de modo que los conductores puedan escoger rutas alternas.
%   \item \textbf{Direccionamiento din\'{a}mico:} similar al anterior, con la
%   variante de presentar a los conductores las gu\'{i}as a rutas alternas.
%   \item \textbf{Sistemas de control de luces de tr\'{a}fico:} manejo de los
%   cambios de luces de los sem\'{a}foros de acuerdo con las necesidades.
% \end{itemize}

\subsection{Sistemas de Control de Sem\'{a}foros}
	Este tipo de sistemas se emplean para coordinar los sem\'{a}foros y otras se\~{n}ales
de tr\'{a}nsito localizados dentro de la red. La comunicaci\'{o}n con estos se logra por
medio de redes que los conectan a una computadora central o red de computadoras
encargadas de administrar el sistema. Para lograr la coordinaci\'{o}n se hace uso de diferentes tipos de implementaciones a trav\'{e}s de t\'{e}cnicas basadas en tiempo o interconexiones cableadas.

	Cabe destacar que el prop\'{o}sito principal de estos sistemas es dar tiempos de
se\~{n}ales favorables para los conductores, adem\'{a}s de proveer acceso a estas
permitiendo a los operadores del sistema controlar y darle mantenimiento. Como
medida adicional, se cuenta con varios mecanismos de detecci\'{o}n o visualizaci\'{o}n de los veh\'{i}culos por medio de sensores o c\'{a}maras de vigilancia.

	Para el funcionamiento de los sem\'{a}foros se poseen dos formas: \textit{pretimed
	operation (operaci\'{o}n con tiempos pre configurados)} y otra
	\textit{actuated operation (operaci\'{o}n accionada)}. Debido a las grandes
	cantidades de veh\'{i}culos que circulan por las calles y a su flujo variable se obtiene un mejor funcionamiento cuando se utiliza el segundo tipo de operaci\'{o}n.

	\textit{Pretimed Operation (Operaci\'{o}n con tiempos pre configurados)}
	corresponde a la forma de operaci\'{o}n en la cual las luces roja, amarilla y verde son temporizadas
en intervalos fijos, debido a que se toman consideraciones sobre los patrones de
tr\'{a}fico al tratar de predecirlos de acuerdo a estimaciones de la hora del
d\'{i}a. No obstante la cantidad de veh\'{i}culos que circulan es muy variable
causando que las predicciones a realizar no sean muy precisas. Por otro lado,
resulta m\'{a}s econ\'{o}mico llevar a cabo estas implementaciones ya que no
requieren de sensores o detectores de tr\'{a}fico en las intersecciones donde se localice alg\'{u}n sem\'{a}foro de la red controlada.

	Por su lado \textit{Actuated Operation (operaci\'{o}n accionado)} consiste
	de controladores de tr\'{a}fico y sensores de veh\'{i}culos localizados en las v\'{i}as que
conducen a las intersecciones donde se han instalado los sem\'{a}foros.  El
algoritmo de control que poseen se refiere principalmente a los intervalos cuando la luz verde ha de cambiar, para los cuales se cuentan con cuatro formas claramente identificadas:

\begin{itemize}
  \item \textbf{Se alcanza el l\'{i}mite de tiempo:} ocurre cuando el tiempo
  m\'{a}ximo indicado por el usuario, es alcanzado.
  \item \textbf{El flujo de tr\'{a}fico cesa  o disminuye:} esta forma se da
  cuando el espacio entre un flujo de tr\'{a}fico y el otro es mayor al umbral determinado por el usuario, el controlador de la se\~{n}al toma la decisi\'{o}n de cambiar la luz verde para favorecer los movimientos en otros sentidos que est\'{e}n demand\'{a}ndolo.
  \item \textbf{Un sem\'{a}foro fuerza la terminaci\'{o}n:} al encontrar dentro
  de un sistema coordinado, el sistema mantiene las sem\'{a}foros funcionando de forma que se realicen los cambios �a prop\'{o}sito� al forzar el cambio de los intervalos de luces verdes.
  \item \textbf{El sem\'{a}foro es adelantado:} ocurre principalmente cuando un
  veh\'{i}culo con prioridad, se aproxima a una intersecci\'{o}n, causando que los intervalos de tiempo de otros sem\'{a}foros se terminen para favorecer  la prioridad de movimiento.
\end{itemize}

	Con estos m\'{e}todos definidos, se puede entender mejor este tipo de operaci\'{o}n.
Para este caso, todos los sem\'{a}foros poseer\'{a}n un intervalo m\'{i}nimo de duraci\'{o}n
para la luz verde, y el tr\'{a}fico en movimiento ser\'{a} atendido mientras se obtengan
registros de los sensores indicando la detecci\'{o}n de veh\'{i}culos
acerc\'{a}ndose, es importante notar que estas aproximaciones deben de ser lo
suficientemente frecuentes para no exceder el limite o umbral establecido que
causa el cambio de la luz. En caso de no suceder esto, los intervalos de
luz verdes seguir\'{a}n extendi\'{e}ndose hasta que todos los veh\'{i}culos
hayan pasado o hasta que se alcance el l\'{i}mite establecido, lo que ocurra
primero.\cite{ResearchAndInnovativeTechnologyAdministration}

	Las formas de operaci\'{o}n mencionadas en este apartado son un factor
importante de entender, ya que estas conforman la base que diferencia el sistema
actual del Centro de Control de Tr\'{a}nsito del  MOPT (Pretimed Operation)
contra un funcionamiento del mismo empleando redes neuronales (similar al Actuated Operation) y a partir de las cuales es posible notar los beneficios que posee una sobre la otra.
